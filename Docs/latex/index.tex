\href{http://rpxnow.com/docs/iphone}{\tt Janrain Engage for iPhone SDK} makes it easy to include third party authentication and social publishing in your iPhone app. This Objective-\/C library includes the same key features as our web version, as well as additional features created specifically for the mobile platform. With as few as three lines of code, you can authenticate your users with their accounts on Google, Yahoo!, Facebook, etc., and they can immediately publish their activities to multiple social networks, including Facebook, Twitter, LinkedIn, MySpace, and Yahoo, through one simple interface.

Beyond authentication and social sharing, the latest release of the Engage for iPhone SDK now allows mobile apps to:
\begin{DoxyItemize}
\item Share content, activities, game scores or invitations via Email or SMS
\item Customize the login experience by displaying native and social login options on the same screen
\item Track popularity and click through rates on various links included in the shared email message with automatic URL shortening for up to 5 URLs
\item Provide an additional level of security with forced re-\/authentication when users are about to make a purchase or conduct a sensitive transaction
\item Configure and maintain separate lists of providers for mobile and web apps
\item Match the look and feel of the iPhone app with customizable background colors, images, and navigation bar tints
\end{DoxyItemize}

Before you begin, you need to have created a \href{https://rpxnow.com/signup_createapp_plus}{\tt Janrain Engage application}, which you can do on \href{http://rpxnow.com}{\tt http://rpxnow.com}

For an overview of how the library works and how you can take advantage of the library's features, please see the \href{http://rpxnow.com/docs/iphone#user_experience}{\tt \char`\"{}Overview\char`\"{}} section of our documentation.

To begin using the SDK, please see the \href{http://rpxnow.com/docs/iphone#quick}{\tt \char`\"{}Quick Start Guide\char`\"{}}.

For more detailed documentation of the library's API, you can use the \href{http://rpxnow.com/docs/iphone_api/annotated.html}{\tt \char`\"{}JREngage API\char`\"{}} documentation. 